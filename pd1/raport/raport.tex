\documentclass[final,12pt]{article}
\usepackage{amsmath}
\usepackage{amssymb}
\usepackage{latexsym}
\usepackage{mathtools}
\usepackage[margin=1in]{geometry}
\usepackage[polish,english]{babel}
\usepackage[utf8]{inputenc}
\usepackage[T1]{fontenc}

\usepackage{url}
\usepackage{xspace}
\usepackage[pdftex]{graphicx}
\usepackage[pdftex]{color}

\usepackage{adjustbox}

\usepackage{listings}

\usepackage{svg}

\usepackage{float}

\usepackage{subfigure}

\usepackage{placeins}

% \usepackage{svg}

% \usepackage{lstautogobble}

% \definecolor{azure}{rgb}{0.0, 0.5, 0.0}
% \definecolor{processblue}{cmyk}{0.96,0,0,0}

\DeclarePairedDelimiter\abs{\lvert}{\rvert}%
\DeclarePairedDelimiter\norm{\lVert}{\rVert}%
\DeclarePairedDelimiter\set{\{}{\}}%
\DeclarePairedDelimiter\tuple{\langle}{\rangle}%

% Swap the definition of \abs* and \norm*, so that \abs
% and \norm resizes the size of the brackets, and the 
% starred version does not.
\makeatletter
\let\oldabs\abs
\def\abs{\@ifstar{\oldabs}{\oldabs*}}
%
\let\oldnorm\norm
\def\norm{\@ifstar{\oldnorm}{\oldnorm*}}
%
\let\oldset\set
\def\set{\@ifstar{\oldset}{\oldset*}}
%
\let\oldtuple\tuple
\def\tuple{\@ifstar{\oldtuple}{\oldtuple*}}
%
\makeatother
%





\definecolor{codegreen}{rgb}{0,0.6,0}
\definecolor{codegray}{rgb}{0.5,0.5,0.5}
\definecolor{codered}{rgb}{0.1,0.1,0.7}
\definecolor{codepurple}{rgb}{0.58,0,0.82}
\definecolor{backcolour}{rgb}{0.95,0.95,0.92}

\lstdefinestyle{mystyle}{
    backgroundcolor=\color{backcolour},   
    commentstyle=\color{codegreen},
    keywordstyle=\color{magenta},
    numberstyle=\tiny\color{codegray},
    stringstyle=\color{codepurple},
    basicstyle=\ttfamily\footnotesize,
    breakatwhitespace=false,         
    breaklines=true,                 
    captionpos=b,                    
    keepspaces=true,                 
    numbers=left,                    
    numbersep=5pt,                  
    showspaces=false,                
    showstringspaces=false,
    showtabs=false,                  
    tabsize=2
}

% mathescape here:
\lstset{style=mystyle,literate={ą}{{\k a}}1
{Ą}{{\k A}}1
{ż}{{\. z}}1
{Ż}{{\. Z}}1
{ź}{{\' z}}1
{Ź}{{\' Z}}1
{ć}{{\' c}}1
{Ć}{{\' C}}1
{ę}{{\k e}}1
{Ę}{{\k E}}1
{ó}{{\' o}}1
{Ó}{{\' O}}1
{ń}{{\' n}}1
{Ń}{{\' N}}1
{ś}{{\' s}}1
{Ś}{{\' S}}1
{ł}{{\l}}1
{Ł}{{\L}}1
{0}{{{\color{codered}{0}}}}1
{1}{{{\color{codered}{1}}}}1
{2}{{{\color{codered}{2}}}}1
{3}{{{\color{codered}{3}}}}1
{4}{{{\color{codered}{4}}}}1
{5}{{{\color{codered}{5}}}}1
{6}{{{\color{codered}{6}}}}1
{7}{{{\color{codered}{7}}}}1
{8}{{{\color{codered}{8}}}}1
{9}{{{\color{codered}{9}}}}1}

% \lstset{language=R,
%     basicstyle=\small\ttfamily,
%     stringstyle=\color{DarkGreen},
%     otherkeywords={0,1,2,3,4,5,6,7,8,9},
%     morekeywords={TRUE,FALSE},
%     deletekeywords={data,frame,length,as,character},
%     keywordstyle=\color{blue},
%     commentstyle=\color{DarkGreen},
% }


\newcommand{\code}{\texttt}

\usepackage{hyperref}
\hypersetup{
    colorlinks,
    citecolor=black,
    filecolor=black,
    linkcolor=black,
    urlcolor=black
}


\newcommand{\figDouble}[3]{
	\FloatBarrier
	\begin{figure}[!htbp]
		\centering
		\subfigure[]{\includesvg[width=0.48\textwidth]{#1}}
		\subfigure[]{\includesvg[width=0.48\textwidth]{#2}}
		\caption{#3}
	\end{figure}
	\FloatBarrier
}

\newcommand{\figSingle}[3]{
	\FloatBarrier
	\begin{figure}[!htbp]
		\centering
		\subfigure[]{\includesvg[width=0.96\textwidth]{#1}}
		\caption{#2}
        \label{#3}
	\end{figure}
	\FloatBarrier
}

\newcommand{\figSinglePNG}[3]{
	\FloatBarrier
	\begin{figure}[!htbp]
		\centering
		\subfigure[]{\includegraphics[width=0.96\textwidth]{#1}}
		\caption{#2}
        \label{#3}
	\end{figure}
	\FloatBarrier
}


\title{Program R w zastosowaniach ekonomicznych i finansowych. Praca domowa 1.}
\author{Andrzej Radzimiński, numer albumu: 429586}
\date{}

\begin{document}

\renewcommand{\contentsname}{Spis treści}

\setlength{\parindent}{0pt}
\noindent

\maketitle

% \tableofcontents

% \vspace{20px}

\section*{Treść zadania (moduł 2)}

Proszę o wczytanie do obiektu np. \textbf{baza} przykładowych danych zawartych w module (na razie
jeszcze tych z lat 2004-2015), a następnie proszę o:

\begin{itemize}
    \item[a)] utworzenie nowego obiektu klasy data frame o nazwie \textbf{baza2} w którym:
    \begin{itemize}
        \item[•] pierwsza „zmienna” (kolumna) to liczby porządkowe (naturalne) od $51$ do $100$
        \item[•] druga „zmienna” ma stałą wartość=20 (można wykorzystać funkcję \code{rep})
        \item[•] trzecia „zmienna” liczba wylosowana z liczb całkowitych z zakresu 1:50 (ze
zwracaniem lub bez -- można wykorzystać funkcję \code{sample(1:50,50))}
        \item[•] czwarta „zmienna” iloczyn pierwiastka kwadratowego pierwszej zmiennej, drugiej
zmiennej oraz kwadratu trzeciej zmiennej.
    \end{itemize}
    \item[b)] wyświetlenie statystyk podsumowujących dla trzeciej i czwartej zmiennej z obiektu \textbf{baza2}.
    \item[c)] wyświetlenie statystyk podsumowujących dla wybranej zmiennej (ze zmiennych o
nazwach X....) z obiektu \textbf{baza} oraz krótką próbę interpretacji (komentarza, co oznacza
    otrzymany wynik) -- ew. można doszukać się błędu w tak zaproponowanej analizie tej zmiennej.
    \item[d)] sprawdzenie i wyświetlenie liczby poziomów dla jednej ze zmiennych powiat, podregion
lub województwo; proszę też wyświetlić 12 pierwszych „poziomów” dla wybranej
zmiennej.
\end{itemize}

\newpage

\section*{Kod}

\begin{lstlisting}

    data_file <- paste(getwd(), "/DaneR.csv", sep="")

    baza <- read.csv(data_file, header=TRUE, sep=";", dec=".", encoding="UTF-8")
    
    len <- 50
    
    baza2 <- data.frame(
        first  = (len+1):(len*2),
        second = rep(20, len),
        third  = sample(1:len, len)
    )
    baza2$forth <- sqrt(baza2$first) * baza2$second * baza2$third**2
    
    summary(baza2$third)
    summary(baza2$forth)
    
    summary(baza$XB14)
    
    nlevels(factor(baza$Powiat))
    
    levels(factor(baza$Powiat))[1:12]
    
\end{lstlisting}

\section*{Wyniki i opisy}

\subsection*{Statystyki podsumowujące dla trzeciej zmiennej obiektu \code{baza}}

\begin{verbatim}
    Min. 1st Qu.  Median    Mean 3rd Qu.    Max. 
    1.00   13.25   25.50   25.50   37.75   50.00 
\end{verbatim}

Trzecia zmienna była losowana bez zwracania z zakresu od $1$ do $50$.
Oznacza to, że wylosowany ciąg liczb musi być premutacją zbioru $\set{1, 2, \ldots, 50}$.  
Wyniki są więc zgodne z oczekiwaniami:
wartość minimalna to $1$, maksymalna to $50$,
mediana oraz średnia to $25.5$.

\subsection*{Statystyki podsumowujące dla czwartej zmiennej obiektu \code{baza}}

\begin{verbatim} 
    Min.  1st Qu.   Median     Mean  3rd Qu.     Max. 
    167.3  31687.1 115341.0 149464.7 238554.8 453019.8 
\end{verbatim}

Czwarta zmienna była obliczana jako iloczyn pierwiastka
kwadratowego pierwszej zmiennej, drugiej zmiennej oraz kwadratu trzeciej zmiennej.

Pierwsza zmienna jest ciągiem liczb naturalnych od $51$ do $100$, czyli jej pierwiastki są
ciągiem monotonicznym od około $7.41$ do $10$.
Druga zmienna jest stała i równa $20$, więc odpowiada ona jedynie za skalowanie wyniku.
Trzecia zmienna jest losowana z zakresu od $1$ do $50$, więc jej kwadrat jest z zakresu od $1$ do $2500$
z okczekiwaną medianą około $25^2 = 625$.

Wyniki są zgodne z oczekiwaniami:
\begin{itemize}
    \item wartość minimalna to $167.3$, w okolicy $20\cdot7.41 = 148.2$,
    \item wartość maksymalna to $453019.8$, w okolicy $20\cdot10\cdot2500 = 500000$,
    \item mediana to $115341.0$, w okolicy $20\cdot8.66\cdot625 = 108250$ ($8.66 \approx \sqrt{75}$ ),
    \item średnia jest znacznie wyższna od mediany, zgodnie rozkłądem zmiennej trzeciej,
    mającej największy względny wpływ na wyniki.
\end{itemize}


\subsection*{Statystyki podsumowujące dla \code{XB14} (liczba ludności na aptekę)}

\begin{verbatim}
   Min. 1st Qu.  Median    Mean 3rd Qu.    Max.    NA's 
   1422    3106    3807    4011    4606   12665       9 
\end{verbatim}

Występuje $9$ pół NA's, co jest związane z $9$ wierszami dotyczącymi ``Powiat m.Wałbrzych'', które
nie posiadają danych.
Można odnaleźć te wiersze np przy użyciu \code{baza[is.na(baza\$XB14),] }


\subsection*{Liczba poziomów dla zmiennej \code{Powiat}}

\begin{verbatim}
    [1] 370
\end{verbatim}

Liczba ta reprezentuje liczbę powiatów w zbiorze danych.
Wartość ta jest zbliżona to faktycznej liczby powaitów w Polsce w roku 2015
(314 powiatów i 66 miast na prawach powiatu).

\subsection*{12 pierwszych poziomów dla zmiennej \code{Powiat}}

\begin{verbatim}
 [1] "Powiat aleksandrowski"      "Powiat augustowski"        
 [3] "Powiat bartoszycki"         "Powiat będziński"          
 [5] "Powiat bełchatowski"        "Powiat białobrzeski"       
 [7] "Powiat białogardzki"        "Powiat białostocki"        
 [9] "Powiat bialski"             "Powiat bielski"            
[11] "Powiat bieruńsko-lędziński" "Powiat bieszczadzki"
\end{verbatim}

Lista ta reprezentuje 12 pierwszych powiatów w zbiorze danych.

\end{document}

