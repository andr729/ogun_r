\documentclass[final,12pt]{article}
\usepackage{amsmath}
\usepackage{amssymb}
\usepackage{latexsym}
\usepackage{mathtools}
\usepackage[margin=1in]{geometry}
\usepackage[polish,english]{babel}
\usepackage[utf8]{inputenc}
\usepackage[T1]{fontenc}

\usepackage{url}
\usepackage{xspace}
\usepackage[pdftex]{graphicx}
\usepackage[pdftex]{color}

\usepackage{adjustbox}

\usepackage{listings}

\usepackage{svg}

\usepackage{float}

\usepackage{subfigure}

\usepackage{placeins}

% \usepackage{svg}

% \usepackage{lstautogobble}

% \definecolor{azure}{rgb}{0.0, 0.5, 0.0}
% \definecolor{processblue}{cmyk}{0.96,0,0,0}

\DeclarePairedDelimiter\abs{\lvert}{\rvert}%
\DeclarePairedDelimiter\norm{\lVert}{\rVert}%
\DeclarePairedDelimiter\set{\{}{\}}%
\DeclarePairedDelimiter\tuple{\langle}{\rangle}%

% Swap the definition of \abs* and \norm*, so that \abs
% and \norm resizes the size of the brackets, and the 
% starred version does not.
\makeatletter
\let\oldabs\abs
\def\abs{\@ifstar{\oldabs}{\oldabs*}}
%
\let\oldnorm\norm
\def\norm{\@ifstar{\oldnorm}{\oldnorm*}}
%
\let\oldset\set
\def\set{\@ifstar{\oldset}{\oldset*}}
%
\let\oldtuple\tuple
\def\tuple{\@ifstar{\oldtuple}{\oldtuple*}}
%
\makeatother
%





\definecolor{codegreen}{rgb}{0,0.6,0}
\definecolor{codegray}{rgb}{0.5,0.5,0.5}
\definecolor{codered}{rgb}{0.1,0.1,0.7}
\definecolor{codepurple}{rgb}{0.58,0,0.82}
\definecolor{backcolour}{rgb}{0.95,0.95,0.92}

\lstdefinestyle{mystyle}{
    backgroundcolor=\color{backcolour},   
    commentstyle=\color{codegreen},
    keywordstyle=\color{magenta},
    numberstyle=\tiny\color{codegray},
    stringstyle=\color{codepurple},
    basicstyle=\ttfamily\footnotesize,
    breakatwhitespace=false,         
    breaklines=true,                 
    captionpos=b,                    
    keepspaces=true,                 
    numbers=left,                    
    numbersep=5pt,                  
    showspaces=false,                
    showstringspaces=false,
    showtabs=false,                  
    tabsize=2
}

% mathescape here:
\lstset{style=mystyle,literate={ą}{{\k a}}1
{Ą}{{\k A}}1
{ż}{{\. z}}1
{Ż}{{\. Z}}1
{ź}{{\' z}}1
{Ź}{{\' Z}}1
{ć}{{\' c}}1
{Ć}{{\' C}}1
{ę}{{\k e}}1
{Ę}{{\k E}}1
{ó}{{\' o}}1
{Ó}{{\' O}}1
{ń}{{\' n}}1
{Ń}{{\' N}}1
{ś}{{\' s}}1
{Ś}{{\' S}}1
{ł}{{\l}}1
{Ł}{{\L}}1
{0}{{{\color{codered}{0}}}}1
{1}{{{\color{codered}{1}}}}1
{2}{{{\color{codered}{2}}}}1
{3}{{{\color{codered}{3}}}}1
{4}{{{\color{codered}{4}}}}1
{5}{{{\color{codered}{5}}}}1
{6}{{{\color{codered}{6}}}}1
{7}{{{\color{codered}{7}}}}1
{8}{{{\color{codered}{8}}}}1
{9}{{{\color{codered}{9}}}}1}

% \lstset{language=R,
%     basicstyle=\small\ttfamily,
%     stringstyle=\color{DarkGreen},
%     otherkeywords={0,1,2,3,4,5,6,7,8,9},
%     morekeywords={TRUE,FALSE},
%     deletekeywords={data,frame,length,as,character},
%     keywordstyle=\color{blue},
%     commentstyle=\color{DarkGreen},
% }


\newcommand{\code}{\texttt}

\usepackage{hyperref}
\hypersetup{
    colorlinks,
    citecolor=black,
    filecolor=black,
    linkcolor=black,
    urlcolor=black
}


\newcommand{\figDouble}[3]{
	\FloatBarrier
	\begin{figure}[!htbp]
		\centering
		\subfigure[]{\includesvg[width=0.48\textwidth]{#1}}
		\subfigure[]{\includesvg[width=0.48\textwidth]{#2}}
		\caption{#3}
	\end{figure}
	\FloatBarrier
}

\newcommand{\figSingle}[3]{
	\FloatBarrier
	\begin{figure}[!htbp]
		\centering
		\subfigure[]{\includesvg[width=0.96\textwidth]{#1}}
		\caption{#2}
        \label{#3}
	\end{figure}
	\FloatBarrier
}

\newcommand{\figSinglePNG}[3]{
	\FloatBarrier
	\begin{figure}[!htbp]
		\centering
		\subfigure[]{\includegraphics[width=0.96\textwidth]{#1}}
		\caption{#2}
        \label{#3}
	\end{figure}
	\FloatBarrier
}


\title{Program R w zastosowaniach ekonomicznych i finansowych. Praca domowa 3.}
\author{Andrzej Radzimiński, numer albumu: 429586}
\date{}

\begin{document}

\renewcommand{\contentsname}{Spis treści}

\setlength{\parindent}{0pt}
\noindent

\maketitle

% \tableofcontents

% \vspace{20px}

\section*{Treść zadania (moduł 8)}

Na danych na których aktualnie pracujemy (na poziomie powiatowym) proszę zbadać czy
statystycznie można uznać, że średnia liczba dzieci na jedno miejsce w przedszkolu w latach
2015 oraz 2014 jest równa. Proszę także zbadać czy można powiedzieć, że średnia liczba
dzieci na jedno miejsce w przedszkolu w 2015 roku była równa średniej z danych za wszystkie
poprzednie dostępne lata (dla tej samej zmiennej). Proszę pamiętać o krótkiej interpretacji. W
razie niejasności zapraszam na forum.

\newpage

\section*{Kod}

\begin{lstlisting}[language=R]
options("width"=100)

data_file <- paste(getwd(), "/dane/DaneR.csv", sep="")
dane <- read.csv(data_file, header=TRUE, sep=";", dec=",", encoding="UTF-8")

# XB03 -- Liczba dzieci na 1 miejsce w przedszkolu

# Porównanie lat 2015 i 2014:

XB03.2015<-as.vector(dane$XB03[dane$Rok==2015])
XB03.2014<-as.vector(dane$XB03[dane$Rok==2014])

summary(XB03.2015)
summary(XB03.2014)

var.test(XB03.2015, XB03.2014)
t.test(XB03.2015, XB03.2014, paired=TRUE, var.equal=TRUE)

# Porównanie lat 2015 i średniej za wszystkie poprzednie lata:

dane_przed_2015 <- dane[dane$Rok<2015,]

XB03.avg <- as.vector(aggregate(
    dane_przed_2015$XB03,
    by=list(dane_przed_2015$Kod_powiat),
    FUN=mean, na.rm=TRUE)$x)

summary(XB03.avg)
var.test(XB03.2015, XB03.avg)
t.test(XB03.2015, XB03.avg, paired=TRUE, var.equal=FALSE)  
\end{lstlisting}

\section*{Wyniki i opisy}

\subsection*{Porównanie średnich z lat 2014 i 2015}

\begin{verbatim}
  Min. 1st Qu.  Median    Mean 3rd Qu.    Max. 
  0.700   1.030   1.250   1.337   1.512   4.520 
   Min. 1st Qu.  Median    Mean 3rd Qu.    Max. 
  0.740   1.060   1.305   1.392   1.590   5.510 


	F test to compare two variances

data:  XB03.2015 and XB03.2014
F = 0.85445, num df = 379, denom df = 379, p-value = 0.1262
alternative hypothesis: true ratio of variances is not equal to 1
95 percent confidence interval:
 0.6984093 1.0453615
sample estimates:
ratio of variances 
         0.8544532 


	Paired t-test

data:  XB03.2015 and XB03.2014
t = -8.8208, df = 379, p-value < 2.2e-16
alternative hypothesis: true mean difference is not equal to 0
95 percent confidence interval:
 -0.06700257 -0.04257638
sample estimates:
mean difference 
    -0.05478947 

\end{verbatim}

Próbki z lat 2014, 2015 mają równe liczebności.
Ze statystyk widać, że są nieznaczne różnice pomiędzy średnimi i medianami,
więc nieznacznie zmienił się kształt rozkładu. 
F test, wykonany za pomocą funkcji \code{var.test}, nie pozwala odrzucić hipotezy zerowej o równości wariancji (p-value powyżej 10\%). 
Przyjmujemy zatem hipotezę o równości wariancji. 

\ \\
Test t, wykonany za pomocą funkcji \code{t.test}, wskazuje na istotną statystycznie różnicę średnich liczby dzieci na jedno miejsce w przedszkolu między latami 2014 i 2015 ($\text{p-value} < 2.2\text{e-}16$). Możemy zatem stwierdzić, że średnia liczba dzieci na jedno miejsce w przedszkolu w 2015 roku była istotnie różna od średniej z 2014 roku.
Wykorzystana została opcja \code{paired=TRUE}, gdyż kolejne obserwacje w
obu próbkach należą do tych samych powiatów. Przyjęto także równe wariancje \code{var.equal=TRUE}, na podstawie wyników testu F.





% TODO: opisując drugi podpunkt poruszyć kwestie z tym, że dla włabrzych nie ma pełnych danych

\end{document}

