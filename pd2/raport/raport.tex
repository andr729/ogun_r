\documentclass[final,12pt]{article}
\usepackage{amsmath}
\usepackage{amssymb}
\usepackage{latexsym}
\usepackage{mathtools}
\usepackage[margin=1in]{geometry}
\usepackage[polish,english]{babel}
\usepackage[utf8]{inputenc}
\usepackage[T1]{fontenc}

\usepackage{url}
\usepackage{xspace}
\usepackage[pdftex]{graphicx}
\usepackage[pdftex]{color}

\usepackage{adjustbox}

\usepackage{listings}

\usepackage{svg}

\usepackage{float}

\usepackage{subfigure}

\usepackage{placeins}

% \usepackage{svg}

% \usepackage{lstautogobble}

% \definecolor{azure}{rgb}{0.0, 0.5, 0.0}
% \definecolor{processblue}{cmyk}{0.96,0,0,0}

\DeclarePairedDelimiter\abs{\lvert}{\rvert}%
\DeclarePairedDelimiter\norm{\lVert}{\rVert}%
\DeclarePairedDelimiter\set{\{}{\}}%
\DeclarePairedDelimiter\tuple{\langle}{\rangle}%

% Swap the definition of \abs* and \norm*, so that \abs
% and \norm resizes the size of the brackets, and the 
% starred version does not.
\makeatletter
\let\oldabs\abs
\def\abs{\@ifstar{\oldabs}{\oldabs*}}
%
\let\oldnorm\norm
\def\norm{\@ifstar{\oldnorm}{\oldnorm*}}
%
\let\oldset\set
\def\set{\@ifstar{\oldset}{\oldset*}}
%
\let\oldtuple\tuple
\def\tuple{\@ifstar{\oldtuple}{\oldtuple*}}
%
\makeatother
%





\definecolor{codegreen}{rgb}{0,0.6,0}
\definecolor{codegray}{rgb}{0.5,0.5,0.5}
\definecolor{codered}{rgb}{0.1,0.1,0.7}
\definecolor{codepurple}{rgb}{0.58,0,0.82}
\definecolor{backcolour}{rgb}{0.95,0.95,0.92}

\lstdefinestyle{mystyle}{
    backgroundcolor=\color{backcolour},   
    commentstyle=\color{codegreen},
    keywordstyle=\color{magenta},
    numberstyle=\tiny\color{codegray},
    stringstyle=\color{codepurple},
    basicstyle=\ttfamily\footnotesize,
    breakatwhitespace=false,         
    breaklines=true,                 
    captionpos=b,                    
    keepspaces=true,                 
    numbers=left,                    
    numbersep=5pt,                  
    showspaces=false,                
    showstringspaces=false,
    showtabs=false,                  
    tabsize=2
}

% mathescape here:
\lstset{style=mystyle,literate={ą}{{\k a}}1
{Ą}{{\k A}}1
{ż}{{\. z}}1
{Ż}{{\. Z}}1
{ź}{{\' z}}1
{Ź}{{\' Z}}1
{ć}{{\' c}}1
{Ć}{{\' C}}1
{ę}{{\k e}}1
{Ę}{{\k E}}1
{ó}{{\' o}}1
{Ó}{{\' O}}1
{ń}{{\' n}}1
{Ń}{{\' N}}1
{ś}{{\' s}}1
{Ś}{{\' S}}1
{ł}{{\l}}1
{Ł}{{\L}}1
{0}{{{\color{codered}{0}}}}1
{1}{{{\color{codered}{1}}}}1
{2}{{{\color{codered}{2}}}}1
{3}{{{\color{codered}{3}}}}1
{4}{{{\color{codered}{4}}}}1
{5}{{{\color{codered}{5}}}}1
{6}{{{\color{codered}{6}}}}1
{7}{{{\color{codered}{7}}}}1
{8}{{{\color{codered}{8}}}}1
{9}{{{\color{codered}{9}}}}1}

% \lstset{language=R,
%     basicstyle=\small\ttfamily,
%     stringstyle=\color{DarkGreen},
%     otherkeywords={0,1,2,3,4,5,6,7,8,9},
%     morekeywords={TRUE,FALSE},
%     deletekeywords={data,frame,length,as,character},
%     keywordstyle=\color{blue},
%     commentstyle=\color{DarkGreen},
% }


\newcommand{\code}{\texttt}

\usepackage{hyperref}
\hypersetup{
    colorlinks,
    citecolor=black,
    filecolor=black,
    linkcolor=black,
    urlcolor=black
}


\newcommand{\figDouble}[3]{
	\FloatBarrier
	\begin{figure}[!htbp]
		\centering
		\subfigure[]{\includesvg[width=0.48\textwidth]{#1}}
		\subfigure[]{\includesvg[width=0.48\textwidth]{#2}}
		\caption{#3}
	\end{figure}
	\FloatBarrier
}

\newcommand{\figSingle}[3]{
	\FloatBarrier
	\begin{figure}[!htbp]
		\centering
		\subfigure[]{\includesvg[width=0.96\textwidth]{#1}}
		\caption{#2}
        \label{#3}
	\end{figure}
	\FloatBarrier
}

\newcommand{\figSinglePNG}[3]{
	\FloatBarrier
	\begin{figure}[!htbp]
		\centering
		\subfigure[]{\includegraphics[width=0.96\textwidth]{#1}}
		\caption{#2}
        \label{#3}
	\end{figure}
	\FloatBarrier
}


\title{Program R w zastosowaniach ekonomicznych i finansowych. Praca domowa 2.}
\author{Andrzej Radzimiński, numer albumu: 429586}
\date{}

\begin{document}

\renewcommand{\contentsname}{Spis treści}

\setlength{\parindent}{0pt}
\noindent

\maketitle

% \tableofcontents

% \vspace{20px}

\section*{Treść zadania (moduł 5)}

\underline{Na danych umieszczonych na kursie}

\begin{itemize}
    \item[a)] proszę wybrać dane dla jednego roku (np. do nowego obiektu), a następnie proszę dla
    każdego województwa przygotować podstawowe podsumowania wybranej zmiennej
    (liczbowej) -- proszę użyć odpowiedniej funkcji, tak żeby podział na województwa był
    automatyczny; w tym punkcie można napisać kilka słów komentarza;

    
    \item[b)] proszę przygotować odpowiednią tabelę pokazującą liczbę powiatów w kontekście
    nakładów inwestycyjnych w przedsiębiorstwach na jednego mieszkańca w
    poszczególnych latach (wskazówka: jedna z danych ma charakter ciągły -- sposób
    poradzenia sobie z tym problemem należy odszukać w przykładach z materiałów);

    \item[c)] proszę przygotować stworzoną w poprzednim punkcie tabelę w wartościach
    procentowych -- jaki wymiar powinien być bazą 100\%?

    \item[d)] proszę sprawdzić (dodając odpowiednie podsumowanie do tabeli z p. b), czy w każdym
    roku jest taka sama liczba powiatów.
\end{itemize}

\underline{Proszę pamiętać o komentarzu do wyników (może być łącznie do punktów b)-d).}

\newpage

\section*{Kod}

\begin{lstlisting}[language=R]
options("width"=100)

data_file <- paste(getwd(), "/dane/DaneR.csv", sep="")
dane <- read.csv(data_file, header=TRUE, sep=";", dec=",", encoding="UTF-8")

# # # # # # # # #
#  Podpunkt a:  #
# # # # # # # # #

# wybrany rok: 2012

dane2012 <- dane[dane$Rok == 2012,]
by(dane2012$XA20, dane2012$Wojewodztwo, summary)


# # # # # # # # #
#  Podpunkt b:  #
# # # # # # # # #

# Mnożymy przez 1e6, aby przeliczyć miliony na jedności:
dane$naklad_na_mieszkanca <- (dane$XA17 * 1e6) / dane$XA29
tabela <- table(cut(dane$naklad_na_mieszkanca, breaks=10), dane$Rok)
tabela


# # # # # # # # #
#  Podpunkt c:  #
# # # # # # # # #

# Wymiar 100% -- sumaryczna liczba województw
round(prop.table(tabela,2),2) 


# # # # # # # # #
#  Podpunkt d:  #
# # # # # # # # #

liczba_lat <- ncol(tabela)
liczba_w = replicate(liczba_lat, 0)
for (i in 1:liczba_lat) {
    liczba_w[i] <- sum(tabela[,i])
}
liczba_w
    
\end{lstlisting}

\section*{Wyniki i opisy}

\subsection{Podpunkt a}

Wybrane został rok 2012. Użyta została funkcja \code{by}, dzięki której możemy
automatycznie uruchomić funkcję \code{summary} dla każdego województwa.
Analizowana była zmienna \code{XA20}, oznaczająca ``Ludność w wieku produkcyjnym''.

Wyniki:

\begin{verbatim}
dane2012$Wojewodztwo: DOLNOŚLĄSKIE
   Min. 1st Qu.  Median    Mean 3rd Qu.    Max.    NA's 
  24768   35948   51368   68120   70726  420646       1 
--------------------------------------------------------------------------- 
dane2012$Wojewodztwo: KUJAWSKO-POMORSKIE
   Min. 1st Qu.  Median    Mean 3rd Qu.    Max. 
  23777   31298   48118   61582   68332  239206 
--------------------------------------------------------------------------- 
dane2012$Wojewodztwo: ŁÓDZKIE
   Min. 1st Qu.  Median    Mean 3rd Qu.    Max. 
  20545   33703   50968   69466   77490  466401 
--------------------------------------------------------------------------- 
dane2012$Wojewodztwo: LUBELSKIE
   Min. 1st Qu.  Median    Mean 3rd Qu.    Max. 
  23700   40760   46912   60038   69594  233130 
--------------------------------------------------------------------------- 
dane2012$Wojewodztwo: LUBUSKIE
   Min. 1st Qu.  Median    Mean 3rd Qu.    Max. 
  24619   35409   44454   49905   64202   84172 
--------------------------------------------------------------------------- 
dane2012$Wojewodztwo: MAŁOPOLSKIE
   Min. 1st Qu.  Median    Mean 3rd Qu.    Max. 
  29156   57796   77972  101877  105684  504357 
--------------------------------------------------------------------------- 
dane2012$Wojewodztwo: MAZOWIECKIE
   Min. 1st Qu.  Median    Mean 3rd Qu.    Max. 
  21046   35828   52504   83287   73258 1104447 
--------------------------------------------------------------------------- 
dane2012$Wojewodztwo: OPOLSKIE
   Min. 1st Qu.  Median    Mean 3rd Qu.    Max. 
  29396   43349   49691   57506   70041   96519 
--------------------------------------------------------------------------- 
dane2012$Wojewodztwo: PODKARPACKIE
   Min. 1st Qu.  Median    Mean 3rd Qu.    Max. 
  15452   41671   49095   57543   74903  124576 
--------------------------------------------------------------------------- 
dane2012$Wojewodztwo: PODLASKIE
   Min. 1st Qu.  Median    Mean 3rd Qu.    Max. 
  13892   29115   34698   47398   44589  202128 
--------------------------------------------------------------------------- 
dane2012$Wojewodztwo: POMORSKIE
   Min. 1st Qu.  Median    Mean 3rd Qu.    Max. 
  24187   44970   60722   76971   79277  303608 
--------------------------------------------------------------------------- 
dane2012$Wojewodztwo: ŚLĄSKIE
   Min. 1st Qu.  Median    Mean 3rd Qu.    Max. 
  35210   56224   83290   86357  106249  204044 
--------------------------------------------------------------------------- 
dane2012$Wojewodztwo: ŚWIĘTOKRZYSKIE
   Min. 1st Qu.  Median    Mean 3rd Qu.    Max. 
  23288   39495   52839   60622   60776  142369 
--------------------------------------------------------------------------- 
dane2012$Wojewodztwo: WARMIŃSKO-MAZURSKIE
   Min. 1st Qu.  Median    Mean 3rd Qu.    Max. 
  16237   29443   40252   47376   61701  119916 
--------------------------------------------------------------------------- 
dane2012$Wojewodztwo: WIELKOPOLSKIE
   Min. 1st Qu.  Median    Mean 3rd Qu.    Max. 
  25184   40270   50062   66747   59590  365653 
--------------------------------------------------------------------------- 
dane2012$Wojewodztwo: ZACHODNIOPOMORSKIE
   Min. 1st Qu.  Median    Mean 3rd Qu.    Max. 
  25961   33758   42586   55964   54403  273480 

\end{verbatim}

Dla województwa Dolnośląskiego pojawiała się pojedyncza wartość \code{NA}, która
jest związana z powiatem M. Wałbrzych, które nie było powiatem w tym roku 2012.

\newpage
\subsection*{Podpunkt b, c}

Wyniki dla 10 równych przedziałów:

\begin{verbatim}
                    2004 2005 2006 2007 2008 2009 2010 2011 2012 2013 2014 2015
(62.8,3.1e+03]         0    0    0    0  291  306  306  283  280  277  245    0
(3.1e+03,6.1e+03]      0    0    0    0   67   51   59   69   75   75   99    0
(6.1e+03,9.11e+03]     0    0    0    0   12   12    6   17   16   19   23    0
(9.11e+03,1.21e+04]    0    0    0    0    6    5    4    5    2    4    5    0
(1.21e+04,1.51e+04]    0    0    0    0    0    1    1    3    2    1    3    0
(1.51e+04,1.81e+04]    0    0    0    0    2    4    2    1    1    0    2    0
(1.81e+04,2.11e+04]    0    0    0    0    0    0    0    1    2    2    1    0
(2.11e+04,2.41e+04]    0    0    0    0    0    0    0    0    1    1    0    0
(2.41e+04,2.71e+04]    0    0    0    0    1    0    1    0    0    0    2    0
(2.71e+04,3.02e+04]    0    0    0    0    0    0    0    0    0    1    0    0
\end{verbatim}

Wyniki z podsumowaniem procentowym:

\begin{verbatim}
                    2004 2005 2006 2007 2008 2009 2010 2011 2012 2013 2014 2015
(62.8,3.1e+03]                          0.77 0.81 0.81 0.75 0.74 0.73 0.64     
(3.1e+03,6.1e+03]                       0.18 0.13 0.16 0.18 0.20 0.20 0.26     
(6.1e+03,9.11e+03]                      0.03 0.03 0.02 0.04 0.04 0.05 0.06     
(9.11e+03,1.21e+04]                     0.02 0.01 0.01 0.01 0.01 0.01 0.01     
(1.21e+04,1.51e+04]                     0.00 0.00 0.00 0.01 0.01 0.00 0.01     
(1.51e+04,1.81e+04]                     0.01 0.01 0.01 0.00 0.00 0.00 0.01     
(1.81e+04,2.11e+04]                     0.00 0.00 0.00 0.00 0.01 0.01 0.00     
(2.11e+04,2.41e+04]                     0.00 0.00 0.00 0.00 0.00 0.00 0.00     
(2.41e+04,2.71e+04]                     0.00 0.00 0.00 0.00 0.00 0.00 0.01     
(2.71e+04,3.02e+04]                     0.00 0.00 0.00 0.00 0.00 0.00 0.00       
\end{verbatim}

Drugi wymiar powinien być bazą 100\%, jako że w ten sposób otrzymuje udział procentowy
danego nakładu w kolejnych latach. Innymi słowy wtedy kolumny sumują się do 100\%.

Możemy zaobserwować brak danych dla lat 2004-2007, 2015.
Dla pozostałych lat możemy zaobserwować stopniowy wzrost nakładów w kolejnych latach, choć
silna kumulacja wyników w pierwszym przedziale sprawia, że dane mogą ukrywać ukryty w tym przedziale trend.

Wyniki z większą granularnością dla pierwszego przedziału (od $0$ do $1000$):

\begin{verbatim}
            2004 2005 2006 2007 2008 2009 2010 2011 2012 2013 2014 2015
(0,100]        0    0    0    0    0    0    0    0    0    1    0    0
(100,200]      0    0    0    0    1    3    2    0    1    1    1    0
(200,300]      0    0    0    0    2    2    2    0    0    0    1    0
(300,400]      0    0    0    0    6    4    4    2    1    0    4    0
(400,500]      0    0    0    0    7   18   10    6    5    8    2    0
(500,600]      0    0    0    0    9   13    9    6    5    5    4    0
(600,700]      0    0    0    0   15   15   18   12    9    5    7    0
(700,800]      0    0    0    0   11   13   15   11   11   12    7    0
(800,900]      0    0    0    0    7   17   14   12   10   14   13    0
(900,1e+03]    0    0    0    0   10   14   12   16   15   13    8    0
\end{verbatim}

Wyniki te potwierdzają hipotezę stopniowego zwiększania nakładów inwestycyjnych na mieszkańca
w kolejnych latach.

\subsection*{Podpunkt d}

\begin{verbatim}
[1]   0   0   0   0 379 379 379 379 379 380 380   0
\end{verbatim}

Wyliczona została liczba powiatów w kolejnych latach, na podstawie danych z tabeli z podpunktu b).
W roku 2013 liczba powiatów wzrosła o 1, co jest zgodne z faktem,
jako że wtedy miasto Wałbrzych stało miastem na prawach powiatu.



\end{document}

